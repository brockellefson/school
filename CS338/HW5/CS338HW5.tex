\documentclass[10pt,a4paper]{article}
\usepackage[utf8]{inputenc}
\usepackage{amsmath}
\usepackage{amsfonts}
\usepackage{amssymb}
\usepackage{graphicx}
\author{Brock Ellefson \\ with collaboration with Elizabeth (Lizzie) Andrews}
\title{CSCI338 HW5}
\begin{document}
\maketitle

\section{Show that ALL$_{DFA}$ is in P}
ALL$_{DFA}$ = $\lbrace$ $<$M$>$ $\mid$ M is a DFA and L(M) = $\sum$$^{\ast}$ $\rbrace$
\\\\
We need to show that ALL$_{DFA}$ is within P. Given any DFA M, we can
determine in polynomial time if M can accept all strings from $\sum$$^{\ast}$. This can be accomplished by g either
a breadth-first or a depth-first search, both of which are proven to be able to run in polynomial
time. If a non-accepting state is reached in M, then L(M)$\neq$ $\sum$$^{\ast}$ and $<$M$>$ is not in ALL$_{DFA}$.\\
\\
Therefore ALL$_{DFA}$ is in P.

\section{Show that ISO $\in$ NP}
G and H can be verified in polynomial time. Let G'
be a reordering of the nodes in G so that they are identical to H. Prove that G' is
identical to H by the following:\\
\\
-Let L be a list that will contain all visited nodes\\
-If the number of nodes in G' is not eqivilent to H, reject.\\
-For each g in G' and each h in H, if each g is equivilent to h, add g to L\\
-If the number of nodes in L is the same as the number of nodes in H and G', accept,
otherwise reject.\\
\\
Therefore a constructor was created that verifies G adn H in polynomial time.\\
Therefore ISO $\in$ NP

\section{SPATH and LPATH}
\subsection{Show that SPATH $\in$ P}
Construct a polynomial time algorithm that decides SPATH.\\
\\
-Place a mark on node a to be the beginning node.\\
-Let i = 0. While i < k, repeat the following step:\\
---For all edges (s, t) in G, if s is marked and t is unmarked, mark t with i + 1.\\
-If node b $\leq$ k, accept. Otherwise, reject.\\

Therefore, SPATH $\in$ P

\subsection{Show that LPATH is NP-Complete}
Prove LPATH $\in$ NP\\
\\
-G = $<$a,b,k$>$\\
-Nondeterministically create path within G thats length is a minimum of k
-If path starts a and ends with b and path only visits a node once, accept. Otherwise, reject
\\
Therefore LPATH $\in$ NP. Onward, prove that every NP problem is reducible to LPATH. Acomplish this by
reduceing the problem UHAMPATH to LPATH.\\
\\
-Construct a formula f' which is a UHAMPATH formula given by f' = f where f =
LPATH $<$G, a, b, k - 1$>$, k being the nodes in G.\\
-f' is in UHAMPATH iff f is in LPATH.
If f' $\in$ UHAMPATH then there exists in G a simple path from a to b that touches all n nodes
in G, by definition of a Hamiltonian Path. Therefore this path n - 1 length.\\
f $\in$ LPATH.\\
\\
If f $\in$ LPATH, then G contains a path of length n - 1 from a to b, by our definition of LPATH.
Since G has n nodes, this means that the path touches every node in G, and it only touches
each node one time. Therefore, the path must be a Hamiltonian Path\\
f' $\in$ UHAMPATH.\\
\\
Therefore LPATH is NP-Complete

\section{Show that DOUBLE-SAT is NP-Complete}
DOUBLE-SAT is a variant of 3SAT, it is in NP.\\
Show that 3SAT is reducible to DOUBLE-SAT.\\
\\
-Construct a DOUBLE-SAT formula f' = f $\wedge$ (x $\vee$ $\neg$x), where f is a 3SAT formula.\\
-f is a satisfying assignment iff f' has two satisfying assignments.\\
-If f has a satisfying assignment u, then f' has two satisfying assignments, either ((u, x) =
false) or ((u, x) = true).\\
-If f' has two satisfying assignments, then the clause (x $\vee$ $\neq$x) has two possible satisfying
assignments, either (x = true) or (x = false).
\\\\
Therefore DOUBLE-SAT is NP-Complete


\section{DOMINATING-SET NP-Complete}
DOMINATING-SET is in NP.\\
Reduce VERTEX-COVER to DOMINATING-SET.\\
\\
-Construct a graph G' such that G' has a dominating set that is size k iff G has a
vertex cover equivilent to k.
-If G' has a dominating set S that is size k, replace any node n in S that has the edge
(u, v) with either of its endpoints, u or v. This is continued until we get a set S' that only contains vertices contained in G.\\
-S' is a vertex cover for G, because for every edge e in the set of edges belonging
to G, e is adjacent to a node in S'.\\
-If a vertex cover S in G that is size k, then S must also be a dominating set for G'.\\
Therefore for any node n in G', either n $\in$ S, or n has some adjacent node that is in S.
Assuming that G is a connected graph and does not contain any isolated vertices, there exists
some edge (n, w) in the set of all edges. \\
\\
Therefore n is either in S or is connected to some other node in S.
Therefore DOMINATING-SET is NP-Complete


\end{document}