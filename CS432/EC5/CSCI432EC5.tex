\documentclass[10pt,a4paper]{article}
\usepackage[utf8]{inputenc}
\usepackage{amsmath}
\usepackage{amsfonts}
\usepackage{amssymb}
\usepackage{graphicx}
\usepackage{setspace}
\author{Brock Ellefson}
\title{CSCI432 EC-05}
\begin{document}
\maketitle
\doublespacing

The CAVE is a interdisciplinary exhibit in Helena that merges the idea of 35,000 year old cave with delta brain wave patterns from upcoming brainwave research technology. This is a massive project with many working parts, and many members of this project come from different fields with different backgrounds in order to tackle the many challenges that a project of this magnitude would present. Being a Computer Scientist, or really any member of the team, would require a lot of patience and communication.

I was part of an interdisciplinary project earlier this semester with teammates coming from different backgrounds, our team was composed of a Civil Engineer, a Mechanical Engineer, myself(CS), and another Computer Science major. Our job was to construct an RC car from scratch and have it traverse through an obstacle course. So each one of us had our own problems to tackle, and then integrate all of our individual work into a final product. So I can only image that this project would have a similar structure to mine, but on a much grander scale. This team really had to utilize the five steps of design thinking (Empathy, Define, Ideate, Prototype, Test). Communicating with all teammates and getting everyone's brainstorming ideas on how to construct this artscience exhibit. As a Computer Scientist, you would have to understand what the defined problem you are working on is and what your role plays. 

\end{document}