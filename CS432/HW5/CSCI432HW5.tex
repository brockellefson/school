\documentclass[10pt,letterpaper]{article}
\usepackage[utf8]{inputenc}
\usepackage{amsmath}
\usepackage{amsfonts}
\usepackage{amssymb}
\usepackage{graphicx}

%\usepackage{algorithm}
%\usepackage{algorithmicx}
%\usepackage{algpseudocode}


\newcommand{\R}{\mathbb{R}}
\renewcommand{\thesubsection}{\thesection.\alph{subsection}}

\author{Brock Ellefson}
\title{CSCI432 HW5}
\begin{document}
\maketitle

\section{Prove that the Frechet distance is a distance metric}

Prove that the Frechet distance is a distance metric.\\
\\
To be a distance metric, you must satisfy 4 requirements:\\
Let X = discrete space\\
A metric is a function d: X x X $\rightarrow$ $\R$ such that:
\begin{enumerate}
  \item d(x,y) = d(y,x)
  \item d(x,y) = 0 $\Leftrightarrow$ x = y
  \item d(x,y) + d(y,z) $\geq$ d(x,z)
  \item d(x,y) $\geq$ 0
\end{enumerate}
So,\\


\section{Recurrence Relations}
\subsection{T(n) = 2T(n/4) + n$^{2}$}
Master's Theorem:\\
T(n) = aT($\frac{n}{b}$) + f(n)\\
Where, a $\geq$ 1, b $>$ 1, f(n) is asymptotically positive\\
\\
T(n) = 2T(n/4) + n$^{2}$\\
a = 2, b = 4, f(n) = n$^{2}$\\
n$^{log_{b}a}$ $\Rightarrow$ n$^{log_{4}2}$ $\Rightarrow$ n$^{1/2}$\\
Case 3:\\
if f(n) is $\Omega$(n$^{log_{b}a + \epsilon}$) for some $\epsilon > 0$ and if f(n/b) $\leq$ f(n), then T(n) = $\theta(f(n))$\\
Therefore, T(n) = $\theta(n^{2})$

\subsection{T(n) = 4T(n/2) + n}
Master's Theorem:\\
T(n) = aT($\frac{n}{b}$) + f(n)\\
Where, a $\geq$ 1, b $>$ 1, f(n) is asymptotically positive\\
\\
T(n) = 4T(n/2) + n\\
a = 4, b = 2, f(n) = n\\
n$^{log_{b}a}$ $\Rightarrow$ n$^{log_{2}4}$ $\Rightarrow$ n$^{2}$\\
Case 1:\\
if f(n) = O(n$^{log_{b}a - \epsilon}$) for some $\epsilon > 0$ then T(n) = $\Theta$( n$^{2}$).\\
Therefore, T(n) =  $\Theta$( n$^{2}$)

\subsection{T(n) = 3T(2n/3) + 4n}
Master's Theorem:\\
T(n) = aT($\frac{n}{b}$) + f(n)\\
Where, a $\geq$ 1, b $>$ 1, f(n) is asymptotically positive\\
\\
T(n) = 3T(2n/3) + 4n\\
a = 3, b = 2/3, f(n) = 4n\\
n$^{log_{b}a}$ $\Rightarrow$ n$^{log_{3/2}3}$ $\Rightarrow$ n$^{2.71}$\\
Case 1:\\
if f(n) = O(n$^{log_{b}a - \epsilon}$) for some $\epsilon > 0$ then T(n) = $\Theta$( n$^{2}$).\\
Therefore, T(n) =  $\Theta$( n$^{2}$)

\subsection{T(n) = T(n/2) + T(n/3)}

\begin{figure}[h]
	\includegraphics[scale = .25]{recurtree1hw5.png}
\end{figure}
\noindent Our longest path in this tree is the rightmost path, following a sequence: $log_{3}n$. So our initial guess is for this recurrence is $O(nlogn)$.\\
\\


\subsection{2T(n/2) + O(log n)}


\section{Climbing Stairs Problem}



\end{document}