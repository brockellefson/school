\documentclass[10pt,letterpaper]{article}
\usepackage[utf8]{inputenc}
\usepackage{amsmath}
\usepackage{amsfonts}
\usepackage{amssymb}
\usepackage{graphicx}
%\usepackage{algorithm}
%\usepackage{algpseudocode}

\newcommand{\R}{\mathbb{R}}
\renewcommand{\thesubsection}{\thesection.\alph{subsection}}

\author{Brock Ellefson, Seth Severa}
\title{CSCI432 GW5}

\begin{document}
\maketitle

\section{Find the Longest Nondecreasing Subsequence, what is the loop invariant? Note: be sure to fully justify.}
Our loop invariant L is: The longest nondecreasing substring A[0..i] is always greater than or equal to the longest nondecreasing substring of A[0..j]

\noindent If a loop invariant is true it must hold true to three conditions:
\begin{enumerate}
  \item $P \Rightarrow L$
  \item $L \wedge X \Rightarrow L$
  \item $L \wedge \sim X \Rightarrow Q$
\end{enumerate}
Where $P$ is our precondition, $X$ is our loop guard, and $Q$ is our postcondition.\\
\\
P implies L because upon entry of the loop, j is 0 and i is 1, which implies that our loop is true.\\
L and X implies L because while in the loop, our base case would be that the longest nondecreasing subsequence is the soley the first element, otherwise, we iterate through the loop. In each iteration, our j value will never be greater than our i value. Therefore A[0..j] will never be greater than A[0..i]\\
L and not X implies Q because once i equals A.size() we break our loop guard X (i $<$ A.size()) causing us to exit the loop with our longest nondecreasing subsequence 
\\
There our loop invariant L is true.

\end{document}
