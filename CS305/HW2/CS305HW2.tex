\documentclass[10pt,letterpaper]{article}
\usepackage[utf8]{inputenc}
\usepackage{amsmath}
\usepackage{amsfonts}
\usepackage{amssymb}
\usepackage{graphicx}
\author{Brock Ellefson}
\title{CSCI 305 HW2}
\begin{document}
\maketitle
\* Discussed ideas with Trent Baker \*
\section{Operational Semantics}

\section{Scoping}
Mark 1:\\
Definition 1: a\\
Definition 2: b,c,d\\\\
Mark 2:\\
Definition 1: a\\
Definition 2: b\\
Definition 3: c,d,e\\\\
Mark 3:\\
Definition 1: a\\
Definition 2: b,c,d\\\\
Mark 4:\\
Definition 1: a,b,c
\section{Conditional Expressions}
k will equal 3 at the end of the loop\\
Because of short-circuit evaluation, in the line:
\begin{verbatim}
if (i % 2 == 0 && foo() % 2 == 0)
\end{verbatim}
foo() will not be executed if i is odd, thus causing j to not be called. This makes k to not increment as frequently. 
\section{Array Representations}

\section{Primitive Data Types}
1. Primitives are not mutable, meaning that you can reassign the variable. This is a really frustrating problem to deal with especially in interpreted languages. \\
2. Undoes polymorphism. In other words, int is required except in special cases, Integer is requested \\
3. Floating types cannot be completely accurate. 
\section{Loops}

\section{Bibliography}
Alpert, S. R. (1998). Primitive Types Considered Harmful [Abstract]. Java Report, 3, 11. Retrieved February 23, 2017, from https://www.research.ibm.com/people/a/alpert/ptch/ptch.html

\end{document}