\documentclass[10pt,letterpaper]{article}
\usepackage[utf8]{inputenc}
\usepackage{amsmath}
\usepackage{amsfonts}
\usepackage{amssymb}
\usepackage{graphicx}

\author{Brock Ellefson}
\title{CSCI466 HW3}
\begin{document}
\maketitle
\section{What is the difference between routing and forwarding?}
The difference between routing and forwarding is that routing is finding an optimal path from the sender to the receiver to send packets. Forwarding is the process of sending packets along the route until it eventually gets to the reciever. 

\section{Describe how packet loss can occur and be eliminated at:}
\subsection{Router input ports?}
It is possible for packet loss to happen at router in ports when multiple ports are using the same out port (head of line blocking).
\\ \\
Ways to avoid head of line blocking:
\begin{enumerate}
  \item Use a switch with multiple packet queues (high priority/low priority) 
  \item Increasing buffer size 
\end{enumerate}

\subsection{Router output ports?}
Router output ports have loss when an excessive queues have  
accumulated, although this can't be avoided, it can be minimized. We can drop a few packets here and there before the buffer reaches capacity, which in the grand scheme of things, is much better than having a mass dropping of packets.


\section{Consider a datagram network using 32-bit host addresses.}
\subsection{Translate  the  above  forwarding  rules  into  a  forwarding  table  that  uses  longest  pre x  matchingto  forward  packets  to  the  correct  interface.   Your  table  should  list  pre x  match  and  forwarding interface} 

\begin{table}[h]
\centering
\caption{Forwarding Table}
\label{my-label}
\begin{tabular}{ll}
destination addr & interface port \\
224.0.0.0/10     & 0              \\
224.64.0.0/16    & 1              \\
224.0.0.0/8      & 2              \\
225.0.0.0/9      & 2              \\
0.0.0.0/0        & 3             
\end{tabular}
\end{table}

\subsection{Describe how the forwarding table determines the outgoing link interface for datagrams addressed to (three different addresses):}
\begin{enumerate}
  \item 200.145.81.85\\
  		It must use interface 3 because it does not match with any other addresses first bits. It doesnt match with the beginning 10 bits of 224.0.0.0 nor the 16 bits of 224.64.0.0, 8 bits of 224.0,0,0, or the 225 address
  		
  		
  \item 255.64.195.60\\
		It must use interface 2 because it matches the 9 bits of the 245.0.0.0 address.  
  
  \item 255.128.17.119\\
  		It must use interface 2 because it matches the 8 bits of the 244.0.0.0 address.
\end{enumerate}

\section{}
\subsection{How a hierarchical organization of the Internet allows it to scale to millions of users?}
\end{document}