\documentclass[10pt,letterpaper]{article}
\usepackage[utf8]{inputenc}
\usepackage{amsmath}
\usepackage{amsfonts}
\usepackage{amssymb}
\usepackage{graphicx}
\author{Brock Ellefson}
\title{CS432 HW2}
\begin{document}
\maketitle
with collaboration : Seth Severa, Yue Hou
\section{Prove that the while loop
in searchFirstOfK in EPI 11.1 (Search a Sorted
Array for First Occurrence of k) is correct.
}

\subsection{What is the pre-condition and post-condition
to the while loop?}
P: left = 0 ; right = A.size() - 1 ; result = -1 \\
Q: left > right ; return result \\

\subsection{What is the loop guard to the while loop?}
X: left $<$= right

\subsection{What is the loop invariant?}
L: K $\leq$ A[0] to left and and K $\geq$ A.size - 1
 
\subsection{Show that P $\Rightarrow$ L}
P $\Rightarrow$ L because if we start with left being the first element and right being the last than K will be always be in between them. 

\subsection{Show that if L is true when you begin an iteration
of the while loop, then L is true at the end
of that iteration.}
If a loop invariant is true it must hold true to three conditions:\\
1. P $\Rightarrow$ L \\
2. L $^{\bigwedge}$ X $\Rightarrow$ L \\
3. L $^{\bigwedge}$ $\sim$ X $\Rightarrow$ Q \\
\\
A base case is when the initial mid is K, because we do not need to iterate through the loop again. L is true here.\\
\\
If K is not mid, we will iterate through the array, decreasing right or increasing left, making the distance between left and right smaller and smaller, however k is still inbetween the left and high. So no matter what our L will always be true.

\subsubsection{Say, in words, what the following statement
means: $\sim$ X $\bigwedge$ L $\rightarrow$ Q  }
The loop terminated, and if L, our loop invariant is true, then the algorithm works.

\subsection{Prove partial correctness}
Since the loop invariant was proved to be true, we proved partial correctness.

\subsection{A decrementing function is a
function such that Y is a well-ordered set,
D(i) is strictly decreasing, and D(i) can be
interpreted as a function value on the i
th iteration
of a loop. What are Y and D in
searchFirstOfK?
}
Y: is the entity of the array
D: right - left 

\subsection{What is the minimum value of Y}
It will be 0, because right - left will get smaller and smaller as the distance between them will get shorter 
\subsection{Show that the loop will terminate in iteration i,
where D(i) is the minimum value of Y}
Well when it right-left hits zero that will be the last iteration of the loop, because our loop guard (left $<$= right) will break the loop and not allow another iteration. 
\end{document}

