\documentclass[10pt,letterpaper]{article}
\usepackage[utf8]{inputenc}
\usepackage{amsmath}
\usepackage{amsfonts}
\usepackage{amssymb}
\usepackage{graphicx}
\author{Brock Ellefson}
\title{WRIT221 Abstract Translation}
\usepackage{setspace}
\doublespacing
\begin{document}

\maketitle

\section*{Original Text}

Education costs, especially textbook prices, are outpacing inflation and becoming a serious financial
burden for college students. Because student textbook use is linked to academic performance,
making textbooks more affordable and accessible is one important way to enhance learning outcomes.
Online booksellers allow student-to- student textbook sales, which can lead to substantial savings for
students. This study demonstrates that the prices of used science textbooks on Amazon.com
demonstrate a clear supply-and- demand pattern, with prices increasing at the beginning of a semester
when supply is low and falling at the end of the semester when supply is high. We show that students
can save 20-33 on science textbooks by timing online textbook purchases carefully. We estimate
savings of 452 over the course of a 4-year college degree by purchasing textbooks early. To enable
students to save on science textbook purchases, we suggest that instructors notify registered students
of the required course materials during the class registration period (near the end of the semester
preceding the class). Such savings can greatly improve the affordability and accessibility
of textbooks for students, which may lead to enhanced learning.

\section*{Facebook Post}
%40 words
Thank god for online student to student textbook sales, I saved like 450 bucks after graduation! The prices that Montana State charges is way too much, I a broke college student, how am I supposed to afford a \$200 textbook? 

\section*{Tweet}
%140 Characters
Yo I got the CS101 textbook for sale on studentsales.com for \$40 bucks.\$\$\$ \#financialwizard. 

\section*{Website Copy}
%200 Word Recreation
Textbook prices are beginning to get out of hand. Many college students can't even afford some or all of their textbooks. Obviously, students who have their textbooks are going to have a better knowledge of the material and a better grade than the average student without a textbook. Clearly, this is a problem that needs to be addressed. Right now, the current strategy is to actually go around the college bookstore and either order textbooks online or from another student at a highly discounted price. Towards the end of the semester, professors should should notify students who plan to take the course the following semester of all the materials needed. This advance will give students who already took the class time to sell their textbooks to upcoming students taking the class at an affordable price. Overall this should create not only a better economy to purchase books and generate a whole new batch of students that will better understand the material, leading to better grades. 

\section*{Analysis}
%Analyize the process of making this posts 
I tried my best to really format the information into a post I would expect to see on sites like Facebook and Twitter. The block of text that was presented for the edit was a fairly simple piece. Pretty easy to understand the material presented, however I found the transition to a Facebook post and Tweet rather awkward. I guess since I don't have a Facebook or a Twitter account anymore I'm not really that used to writing posts like that. Instead of posting an advertisement of promotion of the article about the students buy and sell textbooks, I instead decided to make posts as I student discovering and reacting to the idea. I thought that I would have a better perspective from this angle and therefore making better and more believable posts. I didn't included any images or anything of that nature, I didn't feel like any average joe posting about this would include any images or anything. Again, the summary of the material wasn't too difficult to understand. The way I summarized it in what I believed to be how a college student, such as myself, would talk about it. I wouldn't be too interested in the reports of better grades and gpa, but instead would be drawn to the cheap available textbooks. On the website copy I went on the more technical and professional route. Talking about the studies that show a boost in class grade averages and such, because in my eyes, that would be the more import thing to discuss on a blog or journalism website. Basically, I wanted to structure each of my abstractions to the majority of the audience that would read it. Facebook and Twitter are pretty informal, and posts on there would be casual, whereas people reading an article over the material are looking for something a lot more formal and professional.  

\end{document}