\documentclass[12pt,letterpaper]{article}
\usepackage[utf8]{inputenc}
\usepackage{amsmath}
\usepackage{amsfonts}
\usepackage{amssymb}
\usepackage{graphicx}
\usepackage{setspace}
\author{Brock Ellefson}
\title{Shabout - Destruction, Loss, and Rescue of Modern Iraqi Art }
\begin{document}

\maketitle
\doublespace
Since the invasion of 2003, Iraq has been experiencing an onslaught of continuous acts of destruction that has been dubbed by many historians as deeds of cultural and national cleansing, and not necessarily terrorism. The destruction has been so comprehensive registering the loss that ensued as catastrophic and debilitating. Of particular concern is the loss of modern art in Iraq which seems to be of little interest to the rest of the world. The Islamic State destroyed much of the cultural heritage in the areas under it's control in Iraq. 

About twenty-eight buildings have been looted and destroyed, including Shiite mosques, tombs, shrines and churches. In addition, numerous ancient as well as medieval sites and artifacts have been destroyed. This includes the ancient cities of Hatra and Nimrud, parts of the wall of Ninevah, the ruins of Basha Tapia Castle and Dair Mar Elia, as well as artifacts from the Mosul Museum. In April (2003), the statue of Saddam Hussein was toppled. Years ago Saddam himself dismanted the Unknown Soldier to put up that plaque to celebrate his glory. However, the locals of the area seemed unfazed. Kids were even playing soccer in the street the next day like nothing even happened. Their ideals behind this attitude is that they didn't put him in power, nor did they take him from power, so they have no quail with any of this destruction. 

The Iraqi Museum of Modern Art had a lot of missing and destroyed paintings back in 1980. This includes, but is not limited to, Jewad, Selim, In the Market, Music in the Street, and Faig Hassan's The Tent, and Kurds. However in 2009 the museum was reopened. However there has been a lot of confusion and tension on whether certain paintings are going to be returned and displayed in the museum or not. Faeq Hassan's \textit{Untitled (Salah Al-Din, presumably Battle of Hattin)} carried an estimated \$400,000 to \$500,000. The provenance fot the work says it was in a private collection, having been acquired directly from the artist. The most recent provenance listing in the catalog reads \textit{Property of Private Palestinian Collection}. However, according to reports, Iraqi authorities claim that the work belongs to the state. Maysoon al-Dalumji, an Iraqi Parliament member and head of the commission for culture and information, told the AFP that the oil painting used to be displayed at the ministry of defense but had been stolen, and smuggled out of the country.
\end{document}